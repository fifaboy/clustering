\documentclass[seeding_kmeans]{subfile}

\begin{document}
	\begin{thebibliography}{99}
		\bibitem[Arthur et al.(2007)]{kmeans++} D. Arthur and S. Vassilvitskii. \texttt{k-means++}: The Advantages of Careful Seeding. Proceedings of the eighteenth annual ACM-SIAM symposium on Discrete algorithms. Society for Industrial and Applied Mathematics Philadelphia, PA, USA. pp. $1027–1035$.
		\bibitem[Forgy(1965)]{forgy} E. Forgy. Cluster analysis of multivariate data: Efficiency versus interpretability of classification. Biometrics, $21, 768-780, 1965$.
		\bibitem[Lloyd(1982)]{lloyd} Stuart P. Lloyd. Least squares quantization in pcm. IEEE Transactions on Information Theory, $28(2):129–136, 1982$.
		\bibitem[Dyer(1985)]{dyer} M. E. Dyer. A simple heuristic for the p-center problem. Operations Research Letters, Volume $3$, February $1985$, pp. $285-288$.
		\bibitem[Miligan et al.(1988)]{miligan} G. Milligan, M. C. Coope. A Study of Standardization of Variables in Cluster Analysis, Journal of Classification 5(2) (1988) 181–204.
		\bibitem[Ostrovsky et al.(2006)]{ostrovsky} R. Ostrovsky, Y. Rabani, Leonard J. Schulman, C. Swamy. The Effectiveness of Lloyd-Type Methods for the k-Means Problem. Proceedings of the $47$th Annual Symposium on Foundations of Computer Science. $2006$.
		\bibitem[Celebi et al.(2013)]{celebi} M. Emre Celebi, Hassan A. Kingravi, Patricio A. Vela. A Comparative Study of Efficient Initialization Methods for the K-Means Clustering Algorithm. Expert Systems with Applications, $40(1): 200–210, 2013$.
		\bibitem[Apostol(1976)]{apostol} Tom M. Apostol. Introduction to Analytic Number Theory. Springer, $10.1007/978-3-662-28579-4$, $1976$.
		\bibitem[Pedregosa et al.(2011)]{sklearn} Pedregosa, F., Varoquaux, G., Gramfort, A., Michel, V., Thirion, B., Grisel, O., Blondel, M., Prettenhofer, P., Weiss, R., Dubourg, V., Vanderplas, J., Passos, A., Cournapeau, D., Brucher, M., Perrot, M., Duchesnay, E. Scikit-learn: Machine Learning in Python, , JMLR 12, pp. 2825-2830, 2011.
	\end{thebibliography}
\end{document}